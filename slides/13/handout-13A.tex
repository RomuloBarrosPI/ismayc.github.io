\documentclass[]{article}
\usepackage{lmodern}
\usepackage{amssymb,amsmath}
\usepackage{ifxetex,ifluatex}
\usepackage{fixltx2e} % provides \textsubscript
\ifnum 0\ifxetex 1\fi\ifluatex 1\fi=0 % if pdftex
  \usepackage[T1]{fontenc}
  \usepackage[utf8]{inputenc}
\else % if luatex or xelatex
  \ifxetex
    \usepackage{mathspec}
  \else
    \usepackage{fontspec}
  \fi
  \defaultfontfeatures{Ligatures=TeX,Scale=MatchLowercase}
  \newcommand{\euro}{€}
\fi
% use upquote if available, for straight quotes in verbatim environments
\IfFileExists{upquote.sty}{\usepackage{upquote}}{}
% use microtype if available
\IfFileExists{microtype.sty}{%
\usepackage{microtype}
\UseMicrotypeSet[protrusion]{basicmath} % disable protrusion for tt fonts
}{}
\usepackage[margin=1in]{geometry}
\usepackage{hyperref}
\PassOptionsToPackage{usenames,dvipsnames}{color} % color is loaded by hyperref
\hypersetup{unicode=true,
            pdftitle={Problem Identification Examples},
            pdfborder={0 0 0},
            breaklinks=true}
\urlstyle{same}  % don't use monospace font for urls
\usepackage{graphicx,grffile}
\makeatletter
\def\maxwidth{\ifdim\Gin@nat@width>\linewidth\linewidth\else\Gin@nat@width\fi}
\def\maxheight{\ifdim\Gin@nat@height>\textheight\textheight\else\Gin@nat@height\fi}
\makeatother
% Scale images if necessary, so that they will not overflow the page
% margins by default, and it is still possible to overwrite the defaults
% using explicit options in \includegraphics[width, height, ...]{}
\setkeys{Gin}{width=\maxwidth,height=\maxheight,keepaspectratio}
\setlength{\parindent}{0pt}
\setlength{\parskip}{6pt plus 2pt minus 1pt}
\setlength{\emergencystretch}{3em}  % prevent overfull lines
\providecommand{\tightlist}{%
  \setlength{\itemsep}{0pt}\setlength{\parskip}{0pt}}
\setcounter{secnumdepth}{0}

%%% Use protect on footnotes to avoid problems with footnotes in titles
\let\rmarkdownfootnote\footnote%
\def\footnote{\protect\rmarkdownfootnote}

%%% Change title format to be more compact
\usepackage{titling}

% Create subtitle command for use in maketitle
\newcommand{\subtitle}[1]{
  \posttitle{
    \begin{center}\large#1\end{center}
    }
}

\setlength{\droptitle}{-2em}
  \title{Problem Identification Examples}
  \pretitle{\vspace{\droptitle}\centering\huge}
  \posttitle{\par}
  \author{}
  \preauthor{}\postauthor{}
  \predate{\centering\large\emph}
  \postdate{\par}
  \date{April 25, 2016}


\usepackage{enumerate}
\usepackage{color}
\newenvironment{tight_enumerate}{ \begin{enumerate}[A)] \setlength{\itemsep}{0pt} \setlength{\parskip}{0pt}}{\end{enumerate}}

% Redefines (sub)paragraphs to behave more like sections
\ifx\paragraph\undefined\else
\let\oldparagraph\paragraph
\renewcommand{\paragraph}[1]{\oldparagraph{#1}\mbox{}}
\fi
\ifx\subparagraph\undefined\else
\let\oldsubparagraph\subparagraph
\renewcommand{\subparagraph}[1]{\oldsubparagraph{#1}\mbox{}}
\fi

\begin{document}
\maketitle

\subsection{Identification of Problem
Types}\label{identification-of-problem-types}

Recall the following notation:

\begin{itemize}
\tightlist
\item
  \(K\): categorical variable with 2 groups
\item
  \(G\): categorical variable with 3+ groups
\item
  \(H\): continuous variable
\end{itemize}

For each of the following problems,

\begin{tight_enumerate}
  \item identify the variable(s) and types of variable(s),
  \item identify the model type (e.g., $K_1 \sim K_2$),
  \item determine which type of problem it is 
  \begin{itemize}
    \item One Proportion
    \item Two Proportions
    \item Multiple Proportions (Goodness of Fit)
    \item Multiple Proportions (Test of Independence)
    \item One Mean
    \item Two Means (Independent)
    \item Two Means (Paired)
    \item Multiple Means
    \item Linear Regression
    \item Logistic Regression
  \end{itemize}
  \item draw a sketch of an effective visualization of the sample data and give the name of that type of plot,
  \item write (in symbols) the parameter(s) and point estimate(s),
  \item write the null and alternative hypothesis,
  \item identify what conditions need to be met in order to use a named distribution/theoretical approach,
  \item if relevant, provide a formula for the confidence interval of the parameter of interest,
  \item give a formula for how to calculate the $P$-value based on simulation/randomization AND via the theoretical approach (when applicable), and 
  \item assuming conditions are met, provide the named distribution (e.g., $t(df = 22)$) for the null distribution.
\end{tight_enumerate}

You can assume that the significance level is 5\% for all problems here.

\begin{enumerate}
\def\labelenumi{\arabic{enumi}.}
\tightlist
\item
  The National Survey of Family Growth conducted by the Centers for
  Disease Control gathers information on family life, marriage and
  divorce, pregnancy, infertility, use of contraception, and men's and
  women's health. One of the variables collected on this survey is the
  age at first marriage. 5,534 randomly sampled US women between 2006
  and 2010 completed the survey. The women sampled here had been married
  at least once. Do we have evidence that the mean age that all US women
  from 2006 to 2010 had an average age of first marriage of greater than
  23 years?
\end{enumerate}

\begin{enumerate}
\def\labelenumi{\arabic{enumi}.}
\setcounter{enumi}{1}
\tightlist
\item
  Average income varies from one region of the country to another, and
  it often reflects both lifestyles and regional living expenses.
  Suppose a new graduate is considering a job in two locations,
  Cleveland, OH and Sacramento, CA, and he wants to see whether the
  average income in one of these cities is higher than the other. He
  would like to conduct a hypothesis test based on two small samples
  from the 2000 Census.
\end{enumerate}

\begin{enumerate}
\def\labelenumi{\arabic{enumi}.}
\setcounter{enumi}{2}
\tightlist
\item
  Trace metals in drinking water affect the flavor and an unusually high
  concentration can pose a health hazard. Ten pairs of data were taken
  measuring zinc concentration in bottom water and surface water at 10
  randomly selected locations on a stretch of river. Do the data suggest
  that the true average concentration in the bottom water exceeds that
  of surface water?
\end{enumerate}

\begin{enumerate}
\def\labelenumi{\arabic{enumi}.}
\setcounter{enumi}{3}
\tightlist
\item
  The Child Health and Development Studies investigate a range of
  topics. One study considered a random sample of pregnancies between
  1960 and 1967 among women in the Kaiser Foundation Health Plan in the
  San Francisco East Bay area with the focus on understanding what
  variables tend to influence the baby's \texttt{weight}. The variable
  \texttt{smoke} is coded 1 if the mother is a smoker, and 0 if not.
  Another variable considered is \texttt{parity}, which is 0 if the
  child is the first born, and 1 otherwise. Which one(s) of these
  variables is/are good predictor(s) of the response variable?
\end{enumerate}

\begin{enumerate}
\def\labelenumi{\arabic{enumi}.}
\setcounter{enumi}{4}
\tightlist
\item
  A 2010 survey asked 827 randomly sampled registered voters in
  California ``Do you support? Or do you oppose? Drilling for oil and
  natural gas off the Coast of California? Or do you not know enough to
  say?'' Conduct a hypothesis test to determine if the data provide
  strong evidence that the proportion of college graduates who do not
  have an opinion on this issue is different than that of non-college
  graduates.
\end{enumerate}

\begin{enumerate}
\def\labelenumi{\arabic{enumi}.}
\setcounter{enumi}{5}
\tightlist
\item
  A random sample of 500 U.S. adults were questioned regarding their
  political affiliation (\texttt{democrat} or \texttt{republican}) and
  opinion on a tax reform bill (\texttt{favor}, \texttt{indifferent},
  \texttt{opposed}). Based on this sample, do we have reason to believe
  that political party and opinion on the bill are related?
\end{enumerate}

\begin{enumerate}
\def\labelenumi{\arabic{enumi}.}
\setcounter{enumi}{6}
\tightlist
\item
  A particular brand of candy-coated chocolate comes in five different
  colors: brown, yellow, orange, green, and coffee. The manufacturer of
  the candy says the candies are distributed in the following
  proportions: brown - 40\%, yellow - 20\%, orange = 20\%, and the
  remaining are split evenly between green and coffee. A random sample
  of 580 pieces of this candy are collected. Does this random sample
  provide evidence against the manufacturer's claim?
\end{enumerate}

\begin{enumerate}
\def\labelenumi{\arabic{enumi}.}
\setcounter{enumi}{7}
\tightlist
\item
  On January 28, 1986, a routine launch was anticipated for the
  Challenger space shuttle. Seventy-three seconds into the flight,
  disaster happened: the shuttle broke apart, killing all seven crew
  members on board. An investigation into the cause of the disaster
  focused on a critical seal called an O-ring, and it is believed that
  damage to these O-rings during a shuttle launch may be related to the
  ambient temperature during the launch. Observational data on O-rings
  for 23 randomly selected shuttle missions was collected, where the
  mission order is based on the temperature at the time of the launch.
  \texttt{temp} gives the temperature in Fahrenheit and \texttt{damaged}
  is 1 when O-ring failed and 0 when it was undamaged.
\end{enumerate}

\begin{enumerate}
\def\labelenumi{\arabic{enumi}.}
\setcounter{enumi}{8}
\tightlist
\item
  The CEO of a large electric utility claims that 80 percent of his
  1,000,000 customers are satisfied with the service they receive. To
  test this claim, the local newspaper surveyed 100 customers, using
  simple random sampling. Based on these findings from the sample, can
  we reject the CEO's hypothesis that 80\% of the customers are
  satisfied?
\end{enumerate}

\begin{enumerate}
\def\labelenumi{\arabic{enumi}.}
\setcounter{enumi}{9}
\tightlist
\item
  Chicken farming is a multi-billion dollar industry, and any methods
  that increase the growth rate of young chicks can reduce consumer
  costs while increasing company profits, possibly by millions of
  dollars. An experiment was conducted to measure and compare the
  effectiveness of various feed supplements on the growth rate of
  chickens. Newly hatched chicks were randomly allocated into six
  groups, and each group was given a different feed supplement. Do these
  data provide strong evidence that the average weights of chickens that
  were fed linseed and horsebean are different?
\end{enumerate}

\end{document}
